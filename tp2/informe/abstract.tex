\begin{addmargin}[5em]{5em}
\section*{\centering \runtitulo}

\noindent \textit{k-nearest neighbors} (kNN) es un algoritmo utilizado para, entre otras cosas, la clasificación de entidades en clases. En este trabajo nos proponemos estudiarlo en el contexto de la clasificación de dígitos manuscritos. A medida que avancemos presentaremos la técnica de \textit{principal component analysis} (PCA) para reducir la dimensionalidad de los datos sobre los que entrenar y validar nuestro estimador. Vamos a presentar los valores que maximicen en particular el accuracy para las implementaciones que tenemos de kNN y PCA+kNN. Una vez encontrados estos valores, analizaremos como podemos maximizar la métrica manipulando el balanceo de las instancias de entrenamiento. Finalmente expondremos algunas pruebas sobre cross-validation utilizando K-Fold.

\bigskip

\noindent\textbf{Palabras claves:} kNN, K-Fold, PCA, reconocimiento de dígitos
\end{addmargin}
