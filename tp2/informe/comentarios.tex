\subsection{Relación de PCA con la descomposición SVD}

Sea $\bar{X}$ una muestra centrada en el origen, reducir su dimensionalidad implica encontrar una transformación $P$ tal que $\hat{X}^t=P\bar{X}^t$, siendo $\hat{X}$ tal que la varianza se ve maximizada y su covarianza minimazada.

Sabemos que la matriz de covarianza para $\hat{X}$ se define como

\begin{equation*}
    M_{\hat{X}} = \frac{\hat{X}^t\hat{X}}{n-1} = \frac{P\bar{X}^t\bar{X} P^t}{n-1}
\end{equation*}

Por lo que si consideramos la descomposición SVD de $\bar{X}$, $\bar{X}= U \Sigma V^t $, entonces

\begin{equation*}
    M_{\hat{X}} = (PV) \frac{\Sigma^t \Sigma}{n-1} (PV)^t
\end{equation*}

Tomando $D = \frac{\Sigma^t \Sigma}{n-1}$ con $P = V^T$, tenemos lo que queriamos de $M_{\hat{X}}$.

Además sucede también que $\bar{X} V = U \Sigma$, por lo que $\hat{X}^t = \Sigma^t U^t$.

En conclusión: $P$ es justo lo que queríamos y tiene una forma particular que se deriva de la factorización SVD de $\bar{X}$. $D$ también tiene una forma particular que se deriva de la factorización SVD de $\bar{X}$. A diferencia del método que vimos en clase, en el cual nos basamos en encontrar esta información a partir de $M_{\bar{X}}$ y del hecho que las matrices de covarianza son simétricas.



