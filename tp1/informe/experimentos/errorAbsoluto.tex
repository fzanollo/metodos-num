Como se explicó en [ref a desarrollo o intro] debido a la aritmética finita de la computadora %TODO agregar ref
las operaciones pueden llegar a ser muy inestables \cite{arithmetic}.

Un pequeño ejemplo:

$$\begin{pmatrix}
    5 & -1 & -1 & -1 \\
    -1 & 4 & 0 & -1 \\
    -1 & 0 & 5 & -2 \\
    -1 & -1 & -2 & 6 \\
\end{pmatrix}$$

Luego de correr EG debería quedar:

$$\begin{pmatrix}
    5 & -1 & -1 & -1 \\
    0 & \frac{19}{5} & -\frac{1}{5} & -\frac{6}{5} \\
    0 & 0 & \frac{91}{19} & -\frac{43}{19} \\
    0 & 0 & 0 & \frac{396}{91} \\
\end{pmatrix}$$

Pero por culpa de la falta de precisión de los double queda: %TODO explicar mejor

$$\begin{pmatrix}
    5 & -1 & -1 & -1 \\
    0 & 3.8 & -0.2 & -1.2 \\
    0 & 0 & 4.78947 & -2.26316 \\
    0 & 0 & 0 & 4.35165 \\
\end{pmatrix}$$

Se puede ver que $\frac{396}{91} \cong 4.351648352$ teniendo una diferencia de $0.00000165$ con $4.35165$.
Difieren poco pero es un caso chico, este error se podría repetir y acarrear terminando en un valor en esa posición muy diferente al esperado, afectando así el ranking final.\\

Para ver esto realizamos dos tipos de experimentos, por un lado reutilizamos el batch de tests de la cátedra para ver cuales son las características del error absoluto en un caso promedio.

Por el otro construimos una serie de partidos ficticios con el propósito de que la matriz quede con valores no múltiplos en su diagonal, posiblemente este no sea el peor caso de error pero suponemos que es suficientemente malo. Para calcular el error absoluto, ya que no contamos con el valor esperado, utilizamos la factorización de Cholesky ya que es más estable. Cabe destacar que este método se puede aplicar ya que la matriz resulta simétrica y definida positiva\cite{CMMpaper}.%The Cholesky method is available, because the matrices are not only (obviously) symmetric and real, but are also positive definite.
