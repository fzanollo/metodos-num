Hasta ahora, analizamos CMM desde varias perspectivas.

Desde la perspectiva cuantitativa, comparamos los resultados del método utilizando diferentes matrices $C$, que representan distintas configuraciones de competencias. Llegamos a que comparando los resultados contra otra implementación más estable numéricamente no obtuvimos grandes errores absolutos. Esto puede deberse principalmente a como es una matriz $C$ \textit{de Colley} y que impacto tiene en la estabilidad numérica resolver el sistema mediante Eliminación Gaussiana sin pivoteo, la cual se detalla en la sección \ref{estabilidad_numerica}.

Desde la perspectiva cualitativa, partimos desde el planteo de casos hechos a mano, descubrimos el carácter "transitivo" del método en la sección \ref{conclusion_observaciones} y planteamos una estrategia para aprovecharlo cuando la competencia se da con ciertas características en \ref{estrategia}, entre otras cosas. Comparamos a CMM con Elo y WP mediante el análisis de los resultados de ATP 2015 y llegamos a que los tres devuelven un ranking similar al real y lo utilizamos como punto de partida para estudiar la composición de Elo y WP y contrastarlos con CMM: aunque WP es similar a CMM, sin el sesgo que mencionamos, Elo no y presentamos algunas de sus características. Complementamos esta información mirando el set de datos de NBA 2016 de la cátedra.

¿En qué ámbitos más allá de los que todo el mundo tiene acceso en el día a día podemos utilizar CMM? ¿Podemos utilizarlo para definir una nueva política de reemplazo de caché? Aunque el error absoluto en general es pequeño, ¿es suficiente una implementación como la nuestra para pedir errores aún más chicos? ¿Hubieramos obtenido resultados diferentes si las comparaciones las hubiéramos hecho con otro valor de $k$ para el algoritmo de Elo? Son algunas de las preguntas que quedan abiertas en este trabajo.