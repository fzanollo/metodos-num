\begin{addmargin}[5em]{5em}
\section*{\centering \runtitulo}

\noindent En la actualidad, debido a la importancia y magnitud de los eventos deportivos, a veces es necesario un sistema de rankeo que funcione a pesar de tener pocos encuentros, teniendo en cuenta la dificultad del adversario, pero sin bias previo. Este trabajo trata de la resolución de sistemas de ecuaciones lineales en el contexto específico del \textit{Colley Matrix Method} para ranking de equipos. 
Se busca analizar la utilidad de este método, su estabilidad respecto a los límites de la aritmética finita de las computadoras y su justicia a la hora de rankear equipos. Utilizando como punto de comparación otros métodos como \textit{porcentaje de victorias} (WP) y \textit{Elo rating}, sobre los resultados de partidos de ATP 2015 y NBA 2016.
Se muestra que las características de este método lo hacen justo bajo ciertas normas y el error aritmético generado es chico.

\bigskip

\noindent\textbf{Palabras claves:} Eliminación Gaussiana, Ranking, Colley Matrix Method, Ecuaciones Lineales.
\end{addmargin}