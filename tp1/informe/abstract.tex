\section*{\runtitulo}

\noindent En la actualidad, debido a la importancia y magnitud de los eventos deportivos, aveces es necesario un sistema de rating que funcione a pesar de tener pocos encuentros, teniendo en cuenta dificultad del adversario, pero sin bias previo. Este trabajo trata de resolución de sistemas de ecuaciones lineales en el contexto específico del \textit{Colley Matrix Method} para ranking de equipos. 
Se busca analizar la utilidad de este método, su estabilidad respecto a los límites de la aritmética finita de las computadoras y su justicia a la hora de rankear equipos. Utilizando como punto de comparación otros métodos como \textit{porcentaje de victorias} (WP) y \textit{Elo rating}. %TODO agregar resultados y conclusiones

\bigskip

\noindent\textbf{Palabras claves:} Eliminación Gaussiana, Ranking, Colley Matrix Method, Ecuaciones Lineales.
